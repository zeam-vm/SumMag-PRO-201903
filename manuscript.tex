\documentclass[submit,PRO]{ipsj}
\usepackage{PROpresentation}
\usepackage{caption}
\usepackage{cite}
\usepackage{booktabs} % For formal tables
\usepackage[table,xcdraw]{xcolor}

\usepackage[dvipdfmx]{graphicx}
\usepackage{latexsym}
\usepackage{relsize}
\usepackage{xspace}
\usepackage{jtygm}

\usepackage{color}
\usepackage{fancyvrb}
\newcommand{\VerbBar}{|}
\newcommand{\VERB}{\Verb[commandchars=\\\{\}]}
\DefineVerbatimEnvironment{Highlighting}{Verbatim}{commandchars=\\\{\}}
% Add ',fontsize=\small' for more characters per line
\newenvironment{Shaded}{}{}
\newcommand{\AlertTok}[1]{\textcolor[rgb]{1.00,0.00,0.00}{\textbf{#1}}}
\newcommand{\AnnotationTok}[1]{\textcolor[rgb]{0.38,0.63,0.69}{\textbf{\textit{#1}}}}
\newcommand{\AttributeTok}[1]{\textcolor[rgb]{0.49,0.56,0.16}{#1}}
\newcommand{\BaseNTok}[1]{\textcolor[rgb]{0.25,0.63,0.44}{#1}}
\newcommand{\BuiltInTok}[1]{#1}
\newcommand{\CharTok}[1]{\textcolor[rgb]{0.25,0.44,0.63}{#1}}
\newcommand{\CommentTok}[1]{\textcolor[rgb]{0.38,0.63,0.69}{\textit{#1}}}
\newcommand{\CommentVarTok}[1]{\textcolor[rgb]{0.38,0.63,0.69}{\textbf{\textit{#1}}}}
\newcommand{\ConstantTok}[1]{\textcolor[rgb]{0.53,0.00,0.00}{#1}}
\newcommand{\ControlFlowTok}[1]{\textcolor[rgb]{0.00,0.44,0.13}{\textbf{#1}}}
\newcommand{\DataTypeTok}[1]{\textcolor[rgb]{0.56,0.13,0.00}{#1}}
\newcommand{\DecValTok}[1]{\textcolor[rgb]{0.25,0.63,0.44}{#1}}
\newcommand{\DocumentationTok}[1]{\textcolor[rgb]{0.73,0.13,0.13}{\textit{#1}}}
\newcommand{\ErrorTok}[1]{\textcolor[rgb]{1.00,0.00,0.00}{\textbf{#1}}}
\newcommand{\ExtensionTok}[1]{#1}
\newcommand{\FloatTok}[1]{\textcolor[rgb]{0.25,0.63,0.44}{#1}}
\newcommand{\FunctionTok}[1]{\textcolor[rgb]{0.02,0.16,0.49}{#1}}
\newcommand{\ImportTok}[1]{#1}
\newcommand{\InformationTok}[1]{\textcolor[rgb]{0.38,0.63,0.69}{\textbf{\textit{#1}}}}
\newcommand{\KeywordTok}[1]{\textcolor[rgb]{0.00,0.44,0.13}{\textbf{#1}}}
\newcommand{\NormalTok}[1]{#1}
\newcommand{\OperatorTok}[1]{\textcolor[rgb]{0.40,0.40,0.40}{#1}}
\newcommand{\OtherTok}[1]{\textcolor[rgb]{0.00,0.44,0.13}{#1}}
\newcommand{\PreprocessorTok}[1]{\textcolor[rgb]{0.74,0.48,0.00}{#1}}
\newcommand{\RegionMarkerTok}[1]{#1}
\newcommand{\SpecialCharTok}[1]{\textcolor[rgb]{0.25,0.44,0.63}{#1}}
\newcommand{\SpecialStringTok}[1]{\textcolor[rgb]{0.73,0.40,0.53}{#1}}
\newcommand{\StringTok}[1]{\textcolor[rgb]{0.25,0.44,0.63}{#1}}
\newcommand{\VariableTok}[1]{\textcolor[rgb]{0.10,0.09,0.49}{#1}}
\newcommand{\VerbatimStringTok}[1]{\textcolor[rgb]{0.25,0.44,0.63}{#1}}
\newcommand{\WarningTok}[1]{\textcolor[rgb]{0.38,0.63,0.69}{\textbf{\textit{#1}}}}


\def\newblock{\hskip .11em plus .33em minus .07em}

\def\Underline{\setbox0\hbox\bgroup\let\\\endUnderline}
\def\endUnderline{\vphantom{y}\egroup\smash{\underline{\box0}}\\}
\def\|{\verb|}

\newcommand{\Rplus}{\protect\hspace{-.1em}\protect\raisebox{.8ex}{\smaller{\smaller\smaller\textbf{+}}}}

\newcommand{\Cpp}{\mbox{C\Rplus\Rplus}\xspace}

% キャプション内のレイアウト調整
\makeatletter
\newif\ifcap@minipage
\cap@minipagefalse
\let\minipagebak\minipage
\def\minipage{\cap@minipagetrue\minipagebak}
\let\@captionbak\@caption
\def\@caption{%
 \ifcap@minipage
  \capwidth\hsize\ecapwidth\hsize
 \fi
 \@captionbak}
\makeatother

\PROheadtitle{2018-5-(7): 情報処理学会プログラミング研究会 発表資料 2019年3月19日}

%\setcounter{巻数}{59}%vol53=2012
%\setcounter{号数}{0}
%\setcounter{page}{0}


\def\tightlist{\itemsep1pt\parskip0pt\parsep0pt}

\usepackage[bookmarks=true,dvipdfmx]{hyperref}
\usepackage{pxjahyper}

\begin{document}
\title{SumMag: Elixir マクロのメタプログラミングを用いた並列プログラミング拡張機構 Hastega の解析部の設計と実装}
\etitle{SumMag: Design and Implementation of an Analyzer an Extension Mechanism by Meta-programming Using Elixir Macros}

\affiliate{kitakyu}{北九州市立大学\\
University of Kitakyushu}


\author{山崎 進}{Susumu Yamazaki}{kitakyu}[zacky@kitakyu-u.ac.jp]
\author{久江 雄喜}{Yuki Hisae}{kitakyu}[z8mcb008@gmail.com]

%%% 代表者連絡先(氏名,住所,電話,Fax,電子メールアドレス)
\contact{
山崎 進
北九州市立大学
〒808-0135 北九州市若松区ひびきの1-1
Tel: (093)675-3263, Fax: (093)695-3395
E-mail: zacky@kitakyu-u.ac.jp
}

\begin{abstract}
我々が現在研究開発している並列最適化拡張機構 Hastega(ヘイスガ)は,(1) プログラミング言語
Elixir(エリクサー) で書かれたコード中のパイプライン演算子 と map 関数を用いた記述部分を抽出し,(2)
抽出したコードを並列化・最適化したネイティブコードにコンパイルする.本報告で提案・報告する
SumMag (サムマグ)は,前者(1)を Elixir マクロを用いて実装したメタプログラミングライブラリである.Elixir
は,リストとタプルからなる抽象構文木(AST)を操作するようなメタプログラミング機構,Elixir
マクロを提供しているので,SumMag の実装に用いた.原理としてはパイプライン化された1つ以上の
map 関数の適用をまとめて抽出して関数として分離し,元の Elixir コードの該当部分を抽出した関数への呼び出しに置き換える.このときに
map関数で適用される関数を Elixir コードのレベルでインライン展開し,かつ
map-mapフュージョンを行う.近い将来課題としては,Hastega の残りの機能を実装してファーストリリースとして完成させ
OSS として公開する.
\end{abstract}


\begin{jkeyword}
Elixir,並列プログラミング言語,メタプログラミング
\end{jkeyword}


\begin{eabstract}
A parallel optimization extension mechanism, Hastega, which we are
researching and developing now, (1) extracts description using pipeline
operators and map functions from Elixir program code and (2) compiles
the extracted code into parallelized and optimized native code. SumMag,
which we will propose and report in this report, is a
meta-programming library is a part of the former (1), and is implemented
using Elixir macros. Elixir provides Elixir macros, which is a
meta-programming mechanism that operates abstract syntax trees (ASTs)
consists of lists and tuples, and SumMag is implemented using them.
SumMag works to extract one or more pipe-lined applications of map
functions all together into an separated function, and to replace the
part of the original code into an invocation to the separated function.
At this time, SumMag applies inlining and map-map fusion of the
functions applied in the map functions. In the near future, we will
implement the rest of functions of Hastega, complete it as the first
release, and publish it as an OSS.
\end{eabstract}

\begin{ekeyword}
Elixir, Parallel programming languages, Meta-programming.
\end{ekeyword}

\maketitle

\input{description}

\begin{acknowledgment}
本研究の一部は,北九州産業学術推進機構(略称:FAIS)の新成長戦略推進研究開発事業「シーズ創出・実用性検証事業」の支援を受けた.カラビナテクノロジーの森 正和氏,京都大学の 高瀬 英希先生,デライトシステムズの上野 嘉大氏, rust-jp Slack チームの特に Tatsuya Kawano 氏,Elixir 創始者の Jos\'{e} Valim をはじめとする Lonestar ElixirConf 2019 での発表\cite{ZACKY19-Hastega}で質問いただいた講演者,スタッフ,参加者の皆様,クックパッドの笹田耕一氏をはじめとする第120回情報処理学会プログラミング研究会での発表\cite{ZACKY18J}や第122回情報処理学会プログラミング研究会での発表\cite{Hisae19J}で質疑応答をしていただいた方々,さくらインターネットの松本亮介氏,北陸先端科学技術大学院大学の青木利晃先生,平成30年度の北九州市立大学のディジタルシステム設計と組込みソフトウェアの受講者の方々にはとくに有益な助言を多数いただいた.
\end{acknowledgment}


\bibliographystyle{ipsjsort}
\bibliography{reference}

\end{document}
